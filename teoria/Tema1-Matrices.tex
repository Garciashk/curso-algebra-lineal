% Options for packages loaded elsewhere
\PassOptionsToPackage{unicode}{hyperref}
\PassOptionsToPackage{hyphens}{url}
%
\documentclass[
  ignorenonframetext,
]{beamer}
\usepackage{pgfpages}
\setbeamertemplate{caption}[numbered]
\setbeamertemplate{caption label separator}{: }
\setbeamercolor{caption name}{fg=normal text.fg}
\beamertemplatenavigationsymbolsempty
% Prevent slide breaks in the middle of a paragraph
\widowpenalties 1 10000
\raggedbottom
\setbeamertemplate{part page}{
  \centering
  \begin{beamercolorbox}[sep=16pt,center]{part title}
    \usebeamerfont{part title}\insertpart\par
  \end{beamercolorbox}
}
\setbeamertemplate{section page}{
  \centering
  \begin{beamercolorbox}[sep=12pt,center]{part title}
    \usebeamerfont{section title}\insertsection\par
  \end{beamercolorbox}
}
\setbeamertemplate{subsection page}{
  \centering
  \begin{beamercolorbox}[sep=8pt,center]{part title}
    \usebeamerfont{subsection title}\insertsubsection\par
  \end{beamercolorbox}
}
\AtBeginPart{
  \frame{\partpage}
}
\AtBeginSection{
  \ifbibliography
  \else
    \frame{\sectionpage}
  \fi
}
\AtBeginSubsection{
  \frame{\subsectionpage}
}
\usepackage{amsmath,amssymb}
\usepackage{lmodern}
\usepackage{ifxetex,ifluatex}
\ifnum 0\ifxetex 1\fi\ifluatex 1\fi=0 % if pdftex
  \usepackage[T1]{fontenc}
  \usepackage[utf8]{inputenc}
  \usepackage{textcomp} % provide euro and other symbols
\else % if luatex or xetex
  \usepackage{unicode-math}
  \defaultfontfeatures{Scale=MatchLowercase}
  \defaultfontfeatures[\rmfamily]{Ligatures=TeX,Scale=1}
\fi
% Use upquote if available, for straight quotes in verbatim environments
\IfFileExists{upquote.sty}{\usepackage{upquote}}{}
\IfFileExists{microtype.sty}{% use microtype if available
  \usepackage[]{microtype}
  \UseMicrotypeSet[protrusion]{basicmath} % disable protrusion for tt fonts
}{}
\makeatletter
\@ifundefined{KOMAClassName}{% if non-KOMA class
  \IfFileExists{parskip.sty}{%
    \usepackage{parskip}
  }{% else
    \setlength{\parindent}{0pt}
    \setlength{\parskip}{6pt plus 2pt minus 1pt}}
}{% if KOMA class
  \KOMAoptions{parskip=half}}
\makeatother
\usepackage{xcolor}
\IfFileExists{xurl.sty}{\usepackage{xurl}}{} % add URL line breaks if available
\IfFileExists{bookmark.sty}{\usepackage{bookmark}}{\usepackage{hyperref}}
\hypersetup{
  pdftitle={Tema 1 - Matrices},
  pdfauthor={Juan Gabriel Gomila \& María Santos},
  hidelinks,
  pdfcreator={LaTeX via pandoc}}
\urlstyle{same} % disable monospaced font for URLs
\newif\ifbibliography
\setlength{\emergencystretch}{3em} % prevent overfull lines
\providecommand{\tightlist}{%
  \setlength{\itemsep}{0pt}\setlength{\parskip}{0pt}}
\setcounter{secnumdepth}{-\maxdimen} % remove section numbering
\ifluatex
  \usepackage{selnolig}  % disable illegal ligatures
\fi

\title{Tema 1 - Matrices}
\author{Juan Gabriel Gomila \& María Santos}
\date{null}

\begin{document}
\frame{\titlepage}

\hypertarget{definiciones-generales}{%
\section{Definiciones generales}\label{definiciones-generales}}

\begin{frame}{¿Qué es una matriz?}
\protect\hypertarget{quuxe9-es-una-matriz}{}
Matriz. Sea \((\mathbb{K},+,.)\) un cuerpo conmutativo y \(m,n\ge1\)
enteros. Una matriz \(m\times n\) sobre \(\mathbb{K}\) (o de orden
\(m\times n\) sobre \(\mathbb{K}\)) es una tabla formada por elementos
de \(\mathbb{K}\) dispuestos en \(m\) filas y \(n\) columnas de la forma

\[\begin{pmatrix}a_{11} & a_{12} & \cdots & a_{1n}\\  a_{21} & a_{22} & \cdots & a_{2n}\\ \vdots & \vdots & \ddots & \vdots\\ a_{m1} & a_{m2} & \cdots & a_{mn}\\ \end{pmatrix}\]

con \(a_{ij}\in\mathbb{K};\ i=1,2,\dots,m;\ j=1,2,\dots,n\)
\end{frame}

\begin{frame}{¿Qué es una matriz?}
\protect\hypertarget{quuxe9-es-una-matriz-1}{}
Coeficientes de la matriz. Cada \(a_{ij}\) se denomina término,
coeficiente o entrada de la matriz \(A\). El primer subíndice, \(i\),
indica el número de la fila y el segundo, \(j\), el de la columna que
ocupa el término de la matriz.

\textbf{Ejemplo 1}

\[A = \begin{pmatrix}5 & 0 & 3\\ 9 & 7 & 11\end{pmatrix}\] \(A\) es una
matriz de orden \(2\times 3\) ya que tiene 2 filas y 3 columnas

El elemento \(a_{12}=0\), el elemento \(a_{23}=11\)

\textbf{Ejercicio 1.} ¿Cuáles serían los elementos \(a_{11}\) y
\(a_{13}\)?
\end{frame}

\begin{frame}{¿Dónde están las matrices?}
\protect\hypertarget{duxf3nde-estuxe1n-las-matrices}{}
Conjunto de matrices. Se denotará por
\(\mathcal{M}_{m\times n}(\mathbb{K})\) el conjunto de todas las
matrices de orden \(m \times n\) sobre \(\mathbb{K}\).

Una matriz cualquiera de \(\mathcal{M}_{m\times n}(\mathbb{K})\) se
denotará indistintamente por \(A\), \((a_{ij})_{m\times n}\) o
\((a_{ij})\).

Cuando \(m=n\), el conjunto de todas las matrices de orden
\(n \times n\), \(\mathcal{M}_{n\times n}(\mathbb{K})\), se denota
simplemente por \(\mathcal{M}_{n}(\mathbb{K})\). Las matrices
pertenecientes a este conjunto se dice que son de orden \(n\) en vez de
\(n\times n\).
\end{frame}

\begin{frame}{¿Cuándo dos matrices son iguales?}
\protect\hypertarget{cuuxe1ndo-dos-matrices-son-iguales}{}
Igualdad de matrices. Dadas dos matrices del mismo orden \(m\times n\),
\(A = (a_{ij})_{m\times n}\) y \(B = (b_{ij})_{m\times n}\), son iguales
si

\[a_{ij} = b_{ij}\ \forall i = 1,\dots,m,\ \forall j=1,\dots,n\]

\textbf{Ejemplo 2}

\[A = \begin{pmatrix}3 & 2 & 1\\ 1& 2& 3\end{pmatrix}\quad B=\begin{pmatrix}3 & 1\\ 2 & 2\\ 1 &3\end{pmatrix}\quad C = \begin{pmatrix}3 & 2 & 1\\ 1& 2& 3\end{pmatrix}\quad D = \begin{pmatrix}3 & 2\\ 1& 2\end{pmatrix}\]

\(A\) y \(C\) son las únicas matrices que son iguales

El resto de pares de matrices son diferentes porque tienen órdenes
diferentes: \(A,C\in\mathcal{M}_{2\times 3}(\mathbb{R})\),
\(B\in\mathcal{M}_{3\times 2}(\mathbb{R})\) y
\(D\in\mathcal{M}_{2}(\mathbb{R})\)
\end{frame}

\hypertarget{tipos-de-matrices}{%
\section{Tipos de matrices}\label{tipos-de-matrices}}

\begin{frame}{Tipos de matrices}
\protect\hypertarget{tipos-de-matrices-1}{}
Matriz fila. Se denomina matriz fila a toda matriz que consta de una
única fila
\[A = (a_{11} \ a_{12}\ a_{13}\ \cdots \ a_{1n})\in\mathcal{M}_{1\times n}(\mathbb{K})\]

\textbf{Ejemplo 3}

\[A =\begin{pmatrix}1&-2&3&0&-1&2\end{pmatrix}\in\mathcal{M}_{1\times 6}(\mathbb{R})\]

es una matriz fila
\end{frame}

\begin{frame}{Tipos de matrices}
\protect\hypertarget{tipos-de-matrices-2}{}
Matriz columna. Se denomina matriz columna a toda matriz que consta de
una única columna
\[A = \begin{pmatrix}a_{11} \\ a_{21}\\ \vdots \\ a_{m1}\end{pmatrix}\in\mathcal{M}_{m\times 1}(\mathbb{K})\]

\textbf{Ejemplo 4}

\[A =\begin{pmatrix}1\\0\\2\end{pmatrix}\in\mathcal{M}_{3\times 1}(\mathbb{R})\]

es una matriz columna
\end{frame}

\begin{frame}{Tipos de matrices}
\protect\hypertarget{tipos-de-matrices-3}{}
Matriz nula. Se denota como \(O\) a la matriz nula, matriz con todos sus
coeficientes nulos

\[O = \begin{pmatrix}0&0&\cdots&0\\0&0&\cdots&0\\\ \vdots & \vdots & \ddots& \vdots\\0&0&\cdots&0\end{pmatrix}\in\mathcal{M}_{m\times n}(\mathbb{K})\]
\end{frame}

\begin{frame}{Tipos de matrices}
\protect\hypertarget{tipos-de-matrices-4}{}
Matriz cuadrada. Se denomina matriz cuadrada de orden \(n\) a toda
matriz que consta de \(n\) filas y \(n\) columnas

\[A = \begin{pmatrix}a_{11}&a_{12}&\cdots&a_{1n}\\a_{21}&a_{22}&\cdots&a_{2n}\\\ \vdots & \vdots & \ddots& \vdots\\a_{n1}&a_{n2}&\cdots&a_{nn}\end{pmatrix}\in\mathcal{M}_n(\mathbb{K})\]

\textbf{Ejemplo 5}

\[A =\begin{pmatrix}1 & 2\\0&-1\end{pmatrix}\in\mathcal{M}_2(\mathbb{R})\]

es una matriz cuadrada de orden 2
\end{frame}

\begin{frame}{Matrices cuadradas}
\protect\hypertarget{matrices-cuadradas}{}
Dentro del ámbito de las matrices cuadradas caben las siguientes
definiciones y tipos particulares de matrices:

Diagonal principal. Se denomina diagonal principal de una matriz
cuadrada \(A\) a los elementos \(a_{ii}\) con \(i = 1,\dots, n\).

\[A = \begin{pmatrix}\bf{a_{11}}&a_{12}&\cdots&a_{1n}\\a_{21}&\bf{a_{22}}&\cdots&a_{2n}\\\ \vdots & \vdots & \ddots& \vdots\\a_{n1}&a_{n2}&\cdots&\bf{a_{nn}}\end{pmatrix}\in\mathcal{M}_n(\mathbb{K})\]
\end{frame}

\begin{frame}{Matrices cuadradas}
\protect\hypertarget{matrices-cuadradas-1}{}
Matriz diagonal. Una matriz diagonal es aquella en la cual \(a_{ij}=0\)
siempre que \(i\ne j\)

\[A = \begin{pmatrix}a_{11}&0&\cdots&0\\0&a_{22}&\cdots&0\\\ \vdots & \vdots & \ddots& \vdots\\0&0&\cdots&a_{nn}\end{pmatrix}\in\mathcal{M}_n(\mathbb{K})\]

\textbf{Ejemplo 6}

\[A=\begin{pmatrix}1&0&0\\0&-2&0\\0&0&5\end{pmatrix}\in\mathcal{M}_3(\mathbb{R})\]
es una matriz diagonal de orden 3
\end{frame}

\begin{frame}{Matrices cuadradas}
\protect\hypertarget{matrices-cuadradas-2}{}
Matriz escalar. Una matriz escalar es una matriz diagonal en la cual
\(a_{ii}=\lambda,\ \forall i=1,\dots,n\)

\[A = \begin{pmatrix}\lambda&0&\cdots&0\\0&\lambda&\cdots&0\\\ \vdots & \vdots & \ddots& \vdots\\0&0&\cdots&\lambda\end{pmatrix}\in\mathcal{M}_n(\mathbb{K})\]

\textbf{Ejemplo 7}

\[A=\begin{pmatrix}7&0&0\\0&7&0\\0&0&7\end{pmatrix}\in\mathcal{M}_3(\mathbb{R})\]
es una matriz escalar con escalar \(\lambda=7\in\mathbb{R}\) de orden 3
\end{frame}

\begin{frame}{Matrices cuadradas}
\protect\hypertarget{matrices-cuadradas-3}{}
Matriz identidad. Se denomina matriz unidad o matriz identidad de orden
\(n\), y se denota como \(I_n\) a la matriz escalar en la cual todos los
elementos de la diagonal son \(1\).

\[A = \begin{pmatrix}1&\cdots&0\\ \vdots & \ddots& \vdots\\0&\cdots&1\end{pmatrix}\in\mathcal{M}_n(\mathbb{K})\]

\textbf{Ejemplo 8}

\[I_2=\begin{pmatrix}1&0\\0&1\end{pmatrix}\qquad I_3 = \begin{pmatrix}1&0&0\\0&1&0\\0&0&1\end{pmatrix}\]
son las matrices identidad de orden 2 y 3, respectivamente
\end{frame}

\begin{frame}{Matrices cuadradas}
\protect\hypertarget{matrices-cuadradas-4}{}
Matriz triangular superior. Se denomina matriz triangular superior a
toda matriz en la cual \(a_{ij}=0,\ \forall i>j\). Es decir, todos los
elementos situados por debajo de la diagonal principal son nulos.

\[A = \begin{pmatrix}a_{11}&a_{12}&\cdots&a_{1n}\\0&a_{22}&\cdots&a_{2n}\\\ \vdots & \vdots & \ddots& \vdots\\0&0&\cdots&a_{nn}\end{pmatrix}\in\mathcal{M}_n(\mathbb{K})\]
\end{frame}

\begin{frame}{Matrices cuadradas}
\protect\hypertarget{matrices-cuadradas-5}{}
Matriz triangular inferior. Se denomina matriz triangular inferior a
toda matriz en la cual \(a_{ij}=0,\ \forall i<j\). Es decir, todos los
elementos situados por encima de la diagonal principal son nulos.

\[A = \begin{pmatrix}a_{11}&0&\cdots&0\\a_{21}&a_{22}&\cdots&0\\\ \vdots & \vdots & \ddots& \vdots\\a_{n1}&a_{n2}&\cdots&a_{nn}\end{pmatrix}\in\mathcal{M}_n(\mathbb{K})\]
\end{frame}

\begin{frame}{Matrices cuadradas}
\protect\hypertarget{matrices-cuadradas-6}{}
\textbf{Ejemplo 9}

Esta es una matriz triangular superior de orden 4

\[A = \begin{pmatrix}1&4&-3&2\\0&3&2&5\\0&0&8&-1\\0&0&0&-7\end{pmatrix}\]

y esta es una matriz triangular inferior también de orden 4

\[B = \begin{pmatrix}1&0&0&0\\7&3&0&0\\1&-1&1&0\\5&8&9&3\end{pmatrix}\]
\end{frame}

\begin{frame}{Caso general}
\protect\hypertarget{caso-general}{}
Para matrices en general (no necesariamente cuadradas) se mantendrá la
denominación de matriz triangular superior cuando
\(a_{ij}=0,\forall\ i>j\). Más adelante se estudiarán en profundidad
unos tipos especiales de estas matrices (las matrices escalonadas) que
tendrán una importancia determinante en nuestros estudios.
\end{frame}

\begin{frame}{Caso general}
\protect\hypertarget{caso-general-1}{}
Las matrices triangulares superiores, si no son cuadradas, se
corresponden con los siguientes casos, dependiendo si \(m<n\) o \(n<m\)
respectivamente

\[\begin{pmatrix}a_{11}&a_{12}&\cdots&a_{1m}&\cdots&a_{1n}\\0&a_{22}&\cdots&a_{2m}&\cdots&a_{2n}\\\vdots & \vdots & \ddots& \vdots&\ddots&\vdots\\0&0&\cdots&a_{mm}&\cdots&a_{mn}\end{pmatrix}\quad \begin{pmatrix}a_{11}&a_{12}&\cdots&a_{1n}\\0&a_{22}&\cdots&a_{2n}\\\vdots & \vdots & \ddots& \vdots\\0&0&\cdots&a_{nn}\\0&0&\cdots&0\\\vdots&\vdots&\ddots&\vdots\\0&0&\cdots&0\end{pmatrix}\]
\end{frame}

\begin{frame}{Caso general}
\protect\hypertarget{caso-general-2}{}
\textbf{Ejemplo 10}

Estas son matrices triangulares superiores de orden \(4\times 6\) y
\(5\times 3\) respectivamente

\[A = \begin{pmatrix}1&4&-3&2&3&1\\0&3&2&5&2&-1\\0&0&8&-1&2&-2\\0&0&0&-7&1&3\end{pmatrix}\qquad B = \begin{pmatrix}1&3&-1\\0&3&5\\0&0&1\\0&0&0\\0&0&0\end{pmatrix}\]
\end{frame}

\hypertarget{operaciones-con-matrices}{%
\section{Operaciones con matrices}\label{operaciones-con-matrices}}

\begin{frame}{Operaciones con matrices}
\protect\hypertarget{operaciones-con-matrices-1}{}
Suma de matrices. La suma de dos matrices \(A\) y \(B\) solo es posible
si ambas son del mismo orden \(m\times n\), entonces se suman término a
término. Es decir, dadas \(A=(a_{ij})_{m\times n}\) y
\(B=(b_{ij})_{m\times n}\), se define la suma de \(A\) y \(B\) como la
matriz:

\[C = (c_{ij})_{m\times n}\ \text{ donde}\  c_{ij}=a_{ij}+b_{ij},\\ \forall i=1,\dots,m\ \forall j=1,\dots,n\]
\end{frame}

\begin{frame}{Operaciones con matrices}
\protect\hypertarget{operaciones-con-matrices-2}{}
\textbf{Ejemplo 11}

Sean
\[A = \begin{pmatrix}3&5&-2&0\\0&1&2&-1\\3&2&7&4\end{pmatrix}\qquad B=\begin{pmatrix}1&-4&5&2\\3&2&-4&6\\1&-3&-5&0\end{pmatrix}\]

entonces la suma es

\[A+B = \begin{pmatrix}4&1&3&2\\3&3&-2&5\\4&-1&2&4\end{pmatrix}\]
\end{frame}

\begin{frame}{Operaciones con matrices}
\protect\hypertarget{operaciones-con-matrices-3}{}
Producto por un escalar. Sean \(\lambda\in\mathbb{K}\) y
\(A=(a_{ij})_{m\times n}\in\mathcal{M}_{m\times n}(\mathbb{K})\), se
define el producto \(\lambda A\) como una nueva matriz de orden
\(m\times n\) dada por \[\lambda A = (\lambda\cdot a_{ij})_{m\times n}\]

\textbf{Ejemplo 12}

Dados \(\lambda = 3\) y \(A=\begin{pmatrix}1&2&3\\4&5&6\end{pmatrix}\),
entonces

\[\lambda A=3A = \begin{pmatrix}3&6&9\\12&15&18\end{pmatrix}\]
\end{frame}

\begin{frame}{Operaciones con matrices}
\protect\hypertarget{operaciones-con-matrices-4}{}
Producto de matrices. Para poder realizar el producto de una matriz
\(A\) por una matriz \(B\), el número de columnas de \(A\) ha de
coincidir con el número de filas de \(B\), entonces cada entrada \(ij\)
de la matriz producto se obtiene multiplicando la fila \(i\) de \(A\)
por la columna \(j\) de \(B\) y sumando los números resultantes.
\end{frame}

\begin{frame}{Operaciones con matrices}
\protect\hypertarget{operaciones-con-matrices-5}{}
Concretamente, si \(A\in\mathcal{M}_{m\times n}(\mathbb{K})\) y
\(B\in\mathcal{M}_{n\times p}(\mathbb{K})\), el producto \(AB\) es una
matriz \(C\in\mathcal{M}_{m\times p}(\mathbb{K})\) definida como

\[\begin{pmatrix}a_{11}&a_{12}&a_{13}&\cdots & a_{1n}\\ a_{21}&a_{22}&a_{23}&\cdots & a_{2n}\\\vdots&\vdots&\vdots&\ddots&\vdots\\a_{i1}&a_{i2}&a_{i3}&\cdots & a_{in}\\\vdots&\vdots&\vdots&\ddots&\vdots\\a_{m1}&a_{m2}&a_{m3}&\cdots & a_{mn}\end{pmatrix}\begin{pmatrix}b_{11}&b_{12}&\cdots&b_{1j}&\cdots & b_{1p}\\ b_{21}&b_{22}&\cdots&b_{2j}&\cdots & b_{2p}\\b_{31}&b_{32}&\cdots&b_{3j}&\cdots & b_{3p}\\\vdots&\vdots&\ddots&\vdots&\ddots&\vdots\\b_{n1}&b_{n2}&\cdots&b_{nj}&\cdots & b_{np}\end{pmatrix} = (c_{ij})\]

con
\(c_{ij}=a_{i1}b_{1j}+a_{i2}b_{2j}+\cdots+a_{in}b_{nj}=\sum_{k=1}^na_{ik}b_{kj}\).
Nótese que
\(A_{m\times \textbf{n}}\cdot B_{\textbf{n}\times p}=C_{m\times p}\)
\end{frame}

\begin{frame}{Ejemplo 13}
\protect\hypertarget{ejemplo-13}{}
\textbf{Ejemplo 13}

Dadas
\[A=\begin{pmatrix}-1&2&3&1\\3&-2&1&0\end{pmatrix}\qquad B=\begin{pmatrix}2&1\\0&2\\-1&3\\0&1\end{pmatrix}\]

Entonces, el producto de \(A\) por \(B\) es una matriz cuadrada de orden
2

\[AB =  \begin{pmatrix}-1\cdot 2+2\cdot 0 + 3\cdot(-1)+1\cdot 0&-1\cdot 1+2\cdot2+3\cdot3+1\cdot 1\\3\cdot2-2\cdot0+1\cdot(-1)+0\cdot0&3\cdot1-2\cdot2+1\cdot3+0\cdot1\end{pmatrix} =\begin{pmatrix}-5&13\\5&2\end{pmatrix}\]

mientras que el producto de \(B\) por \(A\) es una matriz de orden 4
\end{frame}

\begin{frame}{Ejemplo 13}
\protect\hypertarget{ejemplo-13-1}{}
\[BA=\begin{pmatrix}2\cdot(-1)+1\cdot3&2\cdot2+1\cdot(-2)&2\cdot3+1\cdot1&2\cdot1+1\cdot0\\0\cdot(-1)+2\cdot3&0\cdot2+2\cdot(-2)&0\cdot3+2\cdot1&0\cdot1+2\cdot0\\-1\cdot(-1)+3\cdot3&-1\cdot2+3\cdot(-2)&-1\cdot3+3\cdot1&-1\cdot1+3\cdot0\\0\cdot(-1)+1\cdot3&0\cdot2+1\cdot(-2)&0\cdot3+1\cdot1&0\cdot1+1\cdot0\end{pmatrix}=\]

\[\begin{pmatrix}1&2&7&2\\6&-4&2&0\\10&-8&0&-1\\3&-2&1&0\end{pmatrix}\]
\end{frame}

\begin{frame}{Operaciones con matrices}
\protect\hypertarget{operaciones-con-matrices-6}{}
Traza. Es la suma de los elementos de la diagonal principal

\[A = \begin{pmatrix}\bf{a_{11}}&a_{12}&\cdots&a_{1n}\\a_{21}&\bf{a_{22}}&\cdots&a_{2n}\\\ \vdots & \vdots & \ddots& \vdots\\a_{n1}&a_{n2}&\cdots&\bf{a_{nn}}\end{pmatrix}\in\mathcal{M}_n(\mathbb{K})\]

\[\text{tr}(A) = a_{11}+\cdots+a_{nn} = \sum_{i = 1}^na_{ii}\]
\end{frame}

\hypertarget{propiedades}{%
\section{Propiedades}\label{propiedades}}

\begin{frame}{Propiedades características}
\protect\hypertarget{propiedades-caracteruxedsticas}{}
Siempre que tengan sentido las operaciones indicadas (es decir, que las
matrices son de los órdenes adecuados para poder realizarlas) se
satisfacen las siguientes propiedades

Propiedad conmutativa. \(A+B=B+A\)

\textbf{Ejemplo 14}

\[A = \begin{pmatrix}2&3&5\\3&2&-1\end{pmatrix}\qquad B=\begin{pmatrix}1&0&-2\\-1&3&6\end{pmatrix}\]

\[A+B = \begin{pmatrix}3&3&3\\2&5&5\end{pmatrix} = B+A\]
\end{frame}

\begin{frame}{Propiedades características}
\protect\hypertarget{propiedades-caracteruxedsticas-1}{}
\textbf{Demostración}

Dadas dos matrices \(A,B\in\mathcal{M}_{m\times n}\), con
\(A = (a_{ij})\), \(B=(b_{ij})\), queremos demostrar que \[A+B = B+A\]

Por un lado, \(A+B = C\) donde \(C = (c_{ij})\) con
\(c_{ij} = a_{ij}+b_{ij}\quad\forall i=1,\dots,m,\ j=1,\dots,n\)

Por otro lado, \(B+A = D\) donde \(D = (d_{ij})\) con
\(d_{ij} = b_{ij}+a_{ij}\quad\forall i=1,\dots,m,\ j=1,\dots,n\)

Pero \(a_{ij}+b_{ij} = b_{ij}+a_{ij}\) ya que
\(a_{ij},b_{ij}\in\mathbb{K}\) con \(\mathbb{K}\) un cuerpo.

Por lo tanto,
\[c_{ij} = d_{ij}\quad\forall i=1,\dots,m,\ j=1,\dots,n\Leftrightarrow C = D \]
\end{frame}

\begin{frame}{Propiedades características}
\protect\hypertarget{propiedades-caracteruxedsticas-2}{}
Propiedad asociativa de la suma. \((A+B)+C=A+(B+C)\)

\textbf{Ejemplo 15}

\[A = \begin{pmatrix}2&3\\3&2\\-1&0\end{pmatrix}\qquad B=\begin{pmatrix}1&0\\3&6\\3&-2\end{pmatrix}\qquad C = \begin{pmatrix}-2&4\\0&5\\7&2\end{pmatrix}\]

\[(A+B)+C = \begin{pmatrix}3&3\\6&8\\2&-2\end{pmatrix}+ \begin{pmatrix}-2&4\\0&5\\7&2\end{pmatrix}= \begin{pmatrix}1&7\\6&13\\9&0\end{pmatrix}=\begin{pmatrix}2&3\\3&2\\-1&0\end{pmatrix}+\begin{pmatrix}-1&4\\3&11\\10&0\end{pmatrix} = A+(B+C)\]
\end{frame}

\begin{frame}{Propiedades características}
\protect\hypertarget{propiedades-caracteruxedsticas-3}{}
\textbf{Demostración}

Dadas las matrices \(A,B,C\in\mathcal{M}_{m\times n}\), con
\(A = (a_{ij})\), \(B=(b_{ij})\), \(C = (c_{ij})\) queremos demostrar
que \[(A+B) + C = A+(B+C)\]

Por un lado, \((A+B) + C = D\) donde \(D = (d_{ij})\) con
\(d_{ij} =( a_{ij}+b_{ij}) +c_{ij}\quad\forall i=1,\dots,m,\ j=1,\dots,n\)

Por otro lado, \(A+(B+C) = E\) donde \(E = (e_{ij})\) con
\(e_{ij} = a_{ij}+(b_{ij}+c_{ij})\quad\forall i=1,\dots,m,\ j=1,\dots,n\)

Pero \((a_{ij}+b_{ij})+c_{ij} = a_{ij}+(b_{ij}+c_{ij})\) ya que
\(a_{ij},b_{ij},c_{ij}\in\mathbb{K}\) con \(\mathbb{K}\) un cuerpo.

Por lo tanto,
\[d_{ij} = e_{ij}\quad\forall i=1,\dots,m,\ j=1,\dots,n\Leftrightarrow D= E \]
\end{frame}

\begin{frame}{Propiedades características}
\protect\hypertarget{propiedades-caracteruxedsticas-4}{}
Elemento neutro de la suma o elemento nulo. \(A+O=O+A=A\)

\textbf{Ejemplo 16}

\[A = \begin{pmatrix}2&3&-1\\-1&0&4\\1&-3&7\end{pmatrix}\]

\[A+O = \begin{pmatrix}2&3&-1\\-1&0&4\\1&-3&7\end{pmatrix}+\begin{pmatrix}0&0&0\\0&0&0\\0&0&0\end{pmatrix}=\begin{pmatrix}2&3&-1\\-1&0&4\\1&-3&7\end{pmatrix}=A\]
\[O+A = \begin{pmatrix}0&0&0\\0&0&0\\0&0&0\end{pmatrix}+\begin{pmatrix}2&3&-1\\-1&0&4\\1&-3&7\end{pmatrix}=\begin{pmatrix}2&3&-1\\-1&0&4\\1&-3&7\end{pmatrix}=A\]
\end{frame}

\begin{frame}{Propiedades características}
\protect\hypertarget{propiedades-caracteruxedsticas-5}{}
\textbf{Demostración}

Dadas las matrices \(A,O\in\mathcal{M}_{m\times n}\), con
\(A = (a_{ij})\) y \(O\) la matriz nula, queremos demostrar que
\[A+O = A\]

Sabemos que, \(A+0 = B\) donde \(B = (b_{ij})\) con
\(b_{ij} = a_{ij}+0\quad\forall i=1,\dots,m,\ j=1,\dots,n\)

Pero \(a_{ij} + 0 = a_{ij}\) ya que \(a_{ij}\in\mathbb{K}\) con
\(\mathbb{K}\) un cuerpo y 0 es el elemento neutro para la suma.

Por lo tanto, \[ A+O= A \]
\end{frame}

\begin{frame}{Propiedades características}
\protect\hypertarget{propiedades-caracteruxedsticas-6}{}
Matriz opuesta. \(\forall\ A=(a_{ij})_{m\times n}\) existe
\(-A = (-a_{ij})_{m\times n}\) tal que \[A+(-A)=(-A)+A=O\]

\textbf{Ejemplo 17}

\[A = \begin{pmatrix}2&3&-1&2\\3&2&5&0\\-1&7&0&4\\-4&1&-3&7\end{pmatrix}\Rightarrow -A = \begin{pmatrix}-2&-3&1&-2\\-3&-2&-5&0\\1&-7&0&-4\\4&-1&3&-7\end{pmatrix}\]

\textbf{Ejercicio 2.} Comprobad que efectivamente se cumple
\[A+(-A)=(-A)+A = O\]
\end{frame}

\begin{frame}{Propiedades características}
\protect\hypertarget{propiedades-caracteruxedsticas-7}{}
\textbf{Demostración}

Dadas las matrices \(A,-A\in\mathcal{M}_{m\times n}\), con
\(A = (a_{ij}),\ -A = (-a_{ij})\), queremos demostrar que \[A+(-A) = 0\]

Sabemos que, \(A+(-A) = B\) donde \(B = (b_{ij})\) con
\(b_{ij} = a_{ij}+(-a_{ij})\quad\forall i=1,\dots,m,\ j=1,\dots,n\)

Pero \(a_{ij} + (-a_{ij}) = 0\) ya que \(a_{ij}\in\mathbb{K}\) con
\(\mathbb{K}\) un cuerpo y \(-a_{ij}\) es el elemento opuesto de
\(a_{ij}\)

Por lo tanto, \[ A+(-A)= 0 \]
\end{frame}

\begin{frame}{Propiedades características}
\protect\hypertarget{propiedades-caracteruxedsticas-8}{}
Propiedad asociativa del producto. \((AB)C=A(BC)\)

Sea
\(A\in\mathcal{M}_{m\times n}(\mathbb{K}),\ B\in\mathcal{M}_{n\times p}(\mathbb{K}),\ C\in\mathcal{M}_{p\times q}(\mathbb{K})\).

Se puede realizar el producto \(AB\), el resultado será una matriz
\(m\times p\) que se podrá multiplicar por \(C\) y el producto \((AB)C\)
será una matriz \(m\times q\).

Análogamente, se puede realizar el producto \(BC\) que dará una matriz
\(n\times q\) y se puede realizar el producto \(A(BC)\) que dará una
matriz \(m\times q\).

Entonces, la propiedad se puede expresar como \[(AB)C=A(BC)\]
\end{frame}

\begin{frame}{Propiedades características}
\protect\hypertarget{propiedades-caracteruxedsticas-9}{}
\textbf{Ejemplo 18}

\[A = \begin{pmatrix}2&3\\5&2\end{pmatrix}\qquad B=\begin{pmatrix}1&0&-1\\2&3&5\end{pmatrix}\qquad C=\begin{pmatrix}2&0\\-1&-2\\0&3\end{pmatrix}\]

\[(AB)C=\begin{pmatrix}8&9&13\\9&6&5\end{pmatrix}\begin{pmatrix}2&0\\-1&-2\\0&3\end{pmatrix}=\begin{pmatrix}7&21\\12&3\end{pmatrix}\]
\[A(BC)=\begin{pmatrix}2&3\\5&2\end{pmatrix}\begin{pmatrix}2&-3\\1&9\end{pmatrix}=\begin{pmatrix}7&21\\12&3\end{pmatrix}\]
\end{frame}

\begin{frame}{Propiedades características}
\protect\hypertarget{propiedades-caracteruxedsticas-10}{}
\textbf{Ejercicio 3}

Se consideran las matrices con coeficientes en \(\mathbb{R}\)

\[A = \begin{pmatrix}1&2\\3&4\end{pmatrix}\quad B = \begin{pmatrix}1&0&1\\0&1&0\end{pmatrix}\quad C = \begin{pmatrix}1&-1\\0&1\\-1&0\end{pmatrix}\]

Probar que \((AB)C = A(BC)\)

¡Atención! La demostración de esta proposición se encuentra en pdf. Este
pdf lo podréis encontrar en el Github, en la carpeta demostraciones, o
bien como material de esta clase.
\end{frame}

\begin{frame}{Propiedades características}
\protect\hypertarget{propiedades-caracteruxedsticas-11}{}
Propiedad distributiva del producto respecto de la suma.
\(A(B+C) = AB+AC\)

\textbf{Ejemplo 19}

\[A = \begin{pmatrix}2&3\\5&2\end{pmatrix}\qquad B=\begin{pmatrix}1&0&-1\\2&3&5\end{pmatrix}\qquad C=\begin{pmatrix}2&0&3\\-1&-2&0\end{pmatrix}\]
\[A(B+C) = \begin{pmatrix}2&3\\5&2\end{pmatrix}\begin{pmatrix}3&0&2\\1&1&5\end{pmatrix} =\begin{pmatrix}9&3&19\\17&2&20\end{pmatrix} \]

\[AB+AC = \begin{pmatrix}8&9&13\\9&6&5\end{pmatrix}+\begin{pmatrix}1&-6&6\\8&-4&15\end{pmatrix} = \begin{pmatrix}9&3&19\\17&2&20\end{pmatrix}\]

\textbf{Ejercicio 4}

Escribe paso a paso la demostración de esta propiedad tomando como
ejemplo las demostraciones anteriores.
\end{frame}

\begin{frame}{Propiedades características}
\protect\hypertarget{propiedades-caracteruxedsticas-12}{}
Elemento neutro del producto o elemento unidad.
\[AI_n = A\qquad I_nB = B\]

Sean \(A\in\mathcal{M}_{m\times n}(\mathbb{K})\) y
\(B\in\mathcal{M}_{n\times p}(\mathbb{K})\).

Se puede realizar el producto \(AI_n\) y el resultado será una matriz
\(m\times n\).

Análogamente, se puede realizar el producto \(I_nB\) y el resultado será
una matriz \(n\times p\).

Además, se puede comprobar que se verifica que
\[AI_n = A\qquad\text{y}\qquad I_nB=B\]
\end{frame}

\begin{frame}{Propiedades características}
\protect\hypertarget{propiedades-caracteruxedsticas-13}{}
\textbf{Ejemplo 20}

\[A = \begin{pmatrix}2&3\\3&2\\-1&0\end{pmatrix}\qquad B=\begin{pmatrix}1&0&3&-1&5\\3&6&-1&-4&2\\3&-2&1&-1&0\end{pmatrix}\]

\[AI_2 = \begin{pmatrix}2&3\\3&2\\-1&0\end{pmatrix}\begin{pmatrix}1&0\\0&1\end{pmatrix} = \begin{pmatrix}2&3\\3&2\\-1&0\end{pmatrix} = A\]

\[I_3B = \begin{pmatrix}1&0&0\\0&1&0\\0&0&1\end{pmatrix}\begin{pmatrix}1&0&3&-1&5\\3&6&-1&-4&2\\3&-2&1&-1&0\end{pmatrix}=\begin{pmatrix}1&0&3&-1&5\\3&6&-1&-4&2\\3&-2&1&-1&0\end{pmatrix}=B\]
\end{frame}

\begin{frame}{Propiedades características}
\protect\hypertarget{propiedades-caracteruxedsticas-14}{}
\textbf{Ejercicio 5}

Considerad las matrices con coeficientes en \(\mathbb{R}\):

\[A=\begin{pmatrix}1&0&1\\0&1&0\end{pmatrix}\quad B = \begin{pmatrix}1&-1\\0&1\\-1&0\end{pmatrix}\]

Probar que:

\[AI_3 = A\] \[I_3B=B\]

\textbf{Ejercicio 6}

Escribe paso a paso la demostración de esta propiedad tomando como
ejemplo las demostraciones anteriores.
\end{frame}

\begin{frame}{Propiedades características}
\protect\hypertarget{propiedades-caracteruxedsticas-15}{}
Observación.Nótese que, en particular, para matrices cuadradas
\(A\in\mathcal{M}_n(\mathbb{K})\), \(I_n\) es un elemento neutro del
producto, es decir: \(AI_n=I_nA=A\) para toda matriz cuadrada \(A\) de
orden \(n\).

\textbf{Ejemplo 21}

\[AI_2 = \begin{pmatrix}2&3\\-1&0\end{pmatrix}\begin{pmatrix}1&0\\0&1\end{pmatrix} = \begin{pmatrix}2&3\\-1&0\end{pmatrix} =\begin{pmatrix}1&0\\0&1\end{pmatrix}\begin{pmatrix}2&3\\-1&0\end{pmatrix}=I_2A\]

\textbf{Ejercicio 7}

Escribe paso a paso la demostración de esta propiedad tomando como
ejemplo las demostraciones anteriores.
\end{frame}

\begin{frame}{Propiedades características}
\protect\hypertarget{propiedades-caracteruxedsticas-16}{}
Propiedad distributiva del producto por escalares respecto de la suma.
\[\lambda(A+B) = \lambda A+\lambda B,\ \lambda\in\mathbb{K}\]

\textbf{Ejercicio 8}

Escribe paso a paso la demostración de esta propiedad tomando como
ejemplo las demostraciones anteriores.
\end{frame}

\begin{frame}{Propiedades características}
\protect\hypertarget{propiedades-caracteruxedsticas-17}{}
\textbf{Ejemplo 22}

\[A = \begin{pmatrix}7&1&5\\-1&-2&6\end{pmatrix},\  B=\begin{pmatrix}2&1&0\\-1&-2&-3\end{pmatrix}\in\mathcal{M}_{2\times 3}(\mathbb{R})\qquad \lambda = 3\in\mathbb{R}\]

\[\lambda(A+B)=3\begin{pmatrix}9&2&5\\-2&-4&3\end{pmatrix}=\begin{pmatrix}27&6&15\\-6&-12&9\end{pmatrix}\]
\[\lambda A+\lambda B=\begin{pmatrix}21&3&15\\-3&-6&18\end{pmatrix}+\begin{pmatrix}6&3&0\\-3&-6&-9\end{pmatrix} =\begin{pmatrix}27&6&15\\-6&-12&9\end{pmatrix} \]
\end{frame}

\begin{frame}{Propiedades características}
\protect\hypertarget{propiedades-caracteruxedsticas-18}{}
Elemento neutro del producto por escalar. \(1A=A\)

\textbf{Ejemplo 23}

\[A = \begin{pmatrix}3&-5&2&0&1\\7&4&1&-3&-2\\6&9&-5&1&0\end{pmatrix}\]

\[1A = \begin{pmatrix}1\cdot3&1\cdot(-5)&1\cdot2&1\cdot0&1\cdot1\\1\cdot7&1\cdot4&1\cdot1&1\cdot(-3)&1\cdot(-2)\\1\cdot6&1\cdot9&1\cdot(-5)&1\cdot1&1\cdot0\end{pmatrix}=\begin{pmatrix}3&-5&2&0&1\\7&4&1&-3&-2\\6&9&-5&1&0\end{pmatrix}=A\]
\end{frame}

\begin{frame}{Propiedades características}
\protect\hypertarget{propiedades-caracteruxedsticas-19}{}
\textbf{Demostración}

Por definición, \[1A = (1\cdot a_{ij}) = (a_{ij}) = A\]

ya que \(a_{ij}\in\mathbb{K}\) y \(1\) es el elemento neutro para el
producto
\end{frame}

\begin{frame}{Propiedades características}
\protect\hypertarget{propiedades-caracteruxedsticas-20}{}
Propiedades distributiva del producto por matrices respecto de la suma
de escalares.

\[(\lambda +\mu)A = \lambda A+\mu A,\ \lambda,\mu\in\mathbb{K}\]

\textbf{Ejercicio 9}

Escribe paso a paso la demostración de esta propiedad tomando como
ejemplo las demostraciones anteriores.
\end{frame}

\begin{frame}{Propiedades características}
\protect\hypertarget{propiedades-caracteruxedsticas-21}{}
\textbf{Ejemplo 24}

\[A =\begin{pmatrix}1&-1\\0&1\end{pmatrix}\qquad \lambda=5,\mu=-7\]

\[(\lambda +\mu)A=-2A = \begin{pmatrix}-2&2\\0&-2\end{pmatrix}\]

\[\lambda A+\mu A=5A+(-7)A=\begin{pmatrix}5&-5\\0&5\end{pmatrix}+\begin{pmatrix}-7&7\\0&-7\end{pmatrix}=\begin{pmatrix}-2&2\\0&-2\end{pmatrix}\]
\end{frame}

\begin{frame}{Propiedades características}
\protect\hypertarget{propiedades-caracteruxedsticas-22}{}
Propiedad asociativa del producto de escalares por una matriz.
\[(\lambda\mu)A = \lambda(\mu A), \ \lambda,\mu\in\mathbb{K}\]

\textbf{Ejercicio 10}

Escribe paso a paso la demostración de esta propiedad tomando como
ejemplo las demostraciones anteriores.
\end{frame}

\begin{frame}{Propiedades características}
\protect\hypertarget{propiedades-caracteruxedsticas-23}{}
\textbf{Ejemplo 25}

\[A =\begin{pmatrix}1&-1\\0&1\end{pmatrix}\qquad \lambda=5,\mu=-7\]
\[(\lambda\mu)A = -35A = \begin{pmatrix}-35&35\\0&-35\end{pmatrix}\]
\[\lambda(\mu A)=5(-7A)=5\begin{pmatrix}-7&7\\0&-7\end{pmatrix}=\begin{pmatrix}-35&35\\0&-35\end{pmatrix}\]
\end{frame}

\begin{frame}{Propiedades características}
\protect\hypertarget{propiedades-caracteruxedsticas-24}{}
Propiedad asociativa del producto de un escalar por dos matrices.
\[\lambda(AB) = (\lambda A)B,\ \lambda\in\mathbb{K}\]

\textbf{Ejemplo 26}
\[A = \begin{pmatrix}2&3\\5&2\end{pmatrix}\qquad B=\begin{pmatrix}1&0&-1\\2&3&5\end{pmatrix}\qquad \lambda=3\]
\[(\lambda A)B=\begin{pmatrix}6&9\\15&6\end{pmatrix}\begin{pmatrix}1&0&-1\\2&3&5\end{pmatrix}=\begin{pmatrix}24&27&39\\27&18&15\end{pmatrix}\]
\[\lambda(AB)=3\cdot\begin{pmatrix}8&9&13\\9&6&5\end{pmatrix}=\begin{pmatrix}24&27&39\\27&18&15\end{pmatrix}\]

\textbf{Ejercicio 11}

Escribe paso a paso la demostración de esta propiedad tomando como
ejemplo las demostraciones anteriores.
\end{frame}

\begin{frame}{Excepciones}
\protect\hypertarget{excepciones}{}
En general, no se cumplen las siguientes propiedades:

Propiedad conmutativa. La multiplicación de matrices no es conmutativa.

\textbf{Ejemplo 27}
\[A = \begin{pmatrix}0&1\\0&0\end{pmatrix},\ B=\begin{pmatrix}0&0\\1&0\end{pmatrix}\in\mathcal{M}_2(\mathbb{R})\]

\[AB = \begin{pmatrix}1&0\\0&0\end{pmatrix}\ne\begin{pmatrix}0&0\\0&1\end{pmatrix}=BA\]
\end{frame}

\begin{frame}{Excepciones}
\protect\hypertarget{excepciones-1}{}
Ley de simplificación. No se cumple la ley de simplificación en un
producto.

\textbf{Ejemplo 28}

\[A = \begin{pmatrix}0&1\\0&2\end{pmatrix},\ B=\begin{pmatrix}1&1\\3&4\end{pmatrix},\ C=\begin{pmatrix}2&5\\3&4\end{pmatrix}\in\mathcal{M}_2(\mathbb{R})\]

satisfacen \(AB=AC\), pero en cambio \(B\ne C\)

\textbf{Ejercicio 12.} Comprueba que efectivamente \(AB=AC\)
\end{frame}

\begin{frame}{Excepciones}
\protect\hypertarget{excepciones-2}{}
Divisores de cero. Existen divisores de 0, es decir
\(AB=0\not\Rightarrow A=0\text{ o }B=0\).

\textbf{Ejemplo 29}

\[A = \begin{pmatrix}0&3\\0&0\end{pmatrix},\ B=\begin{pmatrix}0&2\\0&0\end{pmatrix}\in\mathcal{M}_2(\mathbb{R})\]

pero en cambio \[AB=O=\begin{pmatrix}0&0\\0&0\end{pmatrix}\]
\end{frame}

\begin{frame}{Matrices diagonales y triangulares}
\protect\hypertarget{matrices-diagonales-y-triangulares}{}
Proposición. Sean \(A,B\) dos matrices cuadradas de orden \(n\).

\begin{itemize}
\tightlist
\item
  Si \(A,B\) son matrices diagonales, entonces \(A\) y \(B\) conmutan y
  la matriz producto \(AB=BA\) también es diagonal.
\item
  Si \(A,B\) son matrices triangulares superiores (inferiores) entonces
  el producto \(AB\) es también una matriz triangular superior
  (inferior).
\end{itemize}
\end{frame}

\begin{frame}{Matrices diagonales y triangulares}
\protect\hypertarget{matrices-diagonales-y-triangulares-1}{}
\textbf{Ejercicio 13.} Dadas
\[A=\begin{pmatrix}1&0&0&0\\0&5&0&0\\0&0&-3&0\\0&0&0&2\end{pmatrix}\qquad B=\begin{pmatrix}-2&0&0&0\\0&6&0&0\\0&0&3&0\\0&0&0&9\end{pmatrix}\]
comprobad que \(AB\) y \(BA\) son matrices diagonales.

\textbf{Ejercicio 14.} Dadas
\[A=\begin{pmatrix}1&-1&4&3\\0&5&-2&1\\0&0&-3&7\\0&0&0&2\end{pmatrix}\qquad B=\begin{pmatrix}-2&1&-1&2\\0&6&0&3\\0&0&3&-5\\0&0&0&-2\end{pmatrix}\]
comprobad que \(AB\) y \(BA\) son matrices triangulares superiores.
\end{frame}

\begin{frame}{Matriz transpuesta}
\protect\hypertarget{matriz-transpuesta}{}
Transpuesta de una matriz. Sea
\(A=(a_{ij})_{m\times n}\in\mathcal{M}_{m\times n}(\mathbb{K})\). Se
denomina transpuesta de la matriz \(A\) y se denota como \(A^t\) a la
matriz
\(A^t=(a_{ji})_{n\times m}\in\mathcal{M}_{n\times m}(\mathbb{K})\). Es
decir, la matriz obtenida a partir de \(A\) intercambiando filas por
columnas.

\textbf{Ejemplo 30}

La matriz transpuesta de
\(A = \begin{pmatrix}1&0&3\\2&1&-1\end{pmatrix}\) es
\[A^t=\begin{pmatrix}1&2\\0&1\\3&-1\end{pmatrix}\]
\end{frame}

\begin{frame}{Propiedades matriz transpuesta}
\protect\hypertarget{propiedades-matriz-transpuesta}{}
Entre las propiedades de las matrices transpuestas destacan las
siguientes

Idempotencia. Para toda matriz \(A\), \((A^t)^t = A\).

\textbf{Ejemplo 31}

Teníamos que lLa matriz transpuesta de
\(A = \begin{pmatrix}1&0&3\\2&1&-1\end{pmatrix}\) es
\(A^t=\begin{pmatrix}1&2\\0&1\\3&-1\end{pmatrix}\). Pues la matriz
transpuesta de \(A^t\) es

\[(A^t)^t = \begin{pmatrix}1&0&3\\2&1&-1\end{pmatrix}=A\]

\textbf{Demostración}

\[A\in\mathcal{M}_{m\times n}(\mathbb{K}),\quad A = (a_{ij})\Rightarrow A^t\in\mathcal{M}_{n\times m}(\mathbb{K}),\quad A^t = (a_{ji})\Rightarrow (A^t)^t\in\mathcal{M}_{n\times m}(\mathbb{K}),\quad (A^t)^t = (a_{ij}) = A\]
\end{frame}

\begin{frame}{Propiedades matriz transpuesta}
\protect\hypertarget{propiedades-matriz-transpuesta-1}{}
Transpuesta de una suma. Si \(A\) y \(B\) son matrices del mismo orden
\(m\times n\), entonces \((A+B)^t = A^t+B^t\). Es decir, la transpuesta
de una suma de matrices es la matriz obtenida por la suma de sus
respectivas transpuestas. Además, el resultado se puede generalizar a
\(r\) sumandos y se tiene que si \(A_i\) son todas del mismo orden,
entonces

\[\left(\sum_{i=1}^r A_i\right)^t=\sum_{i=1}^rA_i^t\]

\textbf{Ejercicio 15.} Escribe paso a paso la demostración de esta
propiedad
\end{frame}

\begin{frame}{Propiedades matriz transpuesta}
\protect\hypertarget{propiedades-matriz-transpuesta-2}{}
\textbf{Ejercicio 16.} Comprobar que dadas
\[A = \begin{pmatrix}2&3\\3&2\\-1&0\end{pmatrix}\qquad B=\begin{pmatrix}1&0\\3&6\\3&-2\end{pmatrix}\qquad C = \begin{pmatrix}-2&4\\0&5\\7&2\end{pmatrix}\]
\[(A+B+C)^t=A^t+B^t+C^t\]
\end{frame}

\begin{frame}{Propiedades matriz transpuesta}
\protect\hypertarget{propiedades-matriz-transpuesta-3}{}
Transpuesta de un producto. Si
\(A\in\mathcal{M}_{m\times n}(\mathbb{K})\) y
\(B\in\mathcal{M}_{n\times p}(\mathbb{K})\), entonces la traspuesta del
producto de \(A\) por \(B\) es el producto de las traspuestas pero con
orden cambiado, es decir:
\[(AB)^t=B^tA^t\in\mathcal{M}_{p\times m}(\mathbb{K})\]

\textbf{Ejercicio 17.} Probad que \((AB)^t=B^tA^t\) donde
\[A=\begin{pmatrix}1&0&1\\0&1&0\end{pmatrix}\quad B = \begin{pmatrix}1&-1\\0&1\\-1&0\end{pmatrix}\]

\textbf{Ejercicio 18.} Escribe paso a paso la demostración de esta
propiedad
\end{frame}

\begin{frame}{Matrices cuadradas}
\protect\hypertarget{matrices-cuadradas-7}{}
Nótese que la transposición, en el caso de matrices cuadradas, es una
operación interna

Transposición como operación interna. La transpuesta de una matriz
cuadrada \(A\in\mathcal{M}_n(\mathbb{K})\) es otra matriz cuadrada
\(A^t\in\mathcal{M}_n(\mathbb{K})\)

\textbf{Demostración}

\[A\in\mathcal{M}_n(\mathbb{K})\equiv A\in\mathcal{M}_{n\times n}(\mathbb{K})\Rightarrow A^t\in\mathcal{M}_{n\times n}\]

Por lo tanto, tienen sentido las siguientes definiciones
\end{frame}

\begin{frame}{Matrices cuadradas}
\protect\hypertarget{matrices-cuadradas-8}{}
Sea \(A=(a_{ij})\in\mathcal{M}_n(\mathbb{K})\) una matriz cuadrada

Matriz simétrica. \(A\) es simétrica si coincide con su transpuesta.
Esto causa la simetría de la matriz respecto a su diagonal.

\[A\text{ simétrica }\Leftrightarrow A=A^t\Leftrightarrow a_{ij}=a_{ji}\ \forall i,j\]

\textbf{Ejemplo 32}

\[A = \begin{pmatrix}1&2&3&4\\2&-1&5&6\\3&5&1&7\\4&6&7&-1\end{pmatrix}\]
es una matriz simétrica

\textbf{Ejercicio 19.} Calculad \(A^t\) y veréis que \(A^t=A\).
\end{frame}

\begin{frame}{Matrices cuadradas}
\protect\hypertarget{matrices-cuadradas-9}{}
Matriz antisimétrica. \(A\) es antisimétrica si su transpuesta coincide
con su opuesta, lo cual exige que la diagonal esté compuesta únicamente
por ceros y que los elementos simétricos sean opuestos entre sí.

\[A\text{ antisimétrica }\Leftrightarrow A^t=-A\Leftrightarrow a_{ij}=-a_{ji}\ \forall i,j\]

\textbf{Ejemplo 33}

\[A = \begin{pmatrix}0&2&3&4\\-2&0&5&6\\-3&-5&0&7\\-4&-6&-7&0\end{pmatrix}\]
es una matriz antisimétrica

\textbf{Ejercicio 20.} Calculad \(A^t\) y veréis que \(A^t=-A\).
\end{frame}

\begin{frame}{Matrices cuadradas}
\protect\hypertarget{matrices-cuadradas-10}{}
Matriz regular. \(A\) es invertible o regular si existe otra matriz
cuadrada \(A^{-1}\in\mathcal{M}_n(\mathbb{K})\) tal que
\(AA^{-1}=A^{-1}A=I_n\). Cuando existe esta matriz \(A^{-1}\) es siempre
única, con la propiedad mencionada y se llama matriz inversa de \(A\).

Observación. Nótese que no basta con cumplir solo \(AA^{-1}=I_n\) (o
solo \(A^{-1}A=I_n\)) ya que el producto no es en general conmutativo.
Por tanto, la matriz inversa ha de verificar que los resultados de los
dos productos son la matriz identidad.
\end{frame}

\begin{frame}{Matrices cuadradas}
\protect\hypertarget{matrices-cuadradas-11}{}
\textbf{Ejemplo 34}

\(A = \begin{pmatrix}1&1&0\\0&1&0\\1&0&1\end{pmatrix}\) es una matriz
regular cuya inversa es
\(A^{-1}=\begin{pmatrix}1&-1&0\\0&1&0\\-1&1&1\end{pmatrix}\)

\textbf{Ejercicio 21.} Comprobad que efectivamente
\(AA^{-1}=A^{-1}A=I_3\)
\end{frame}

\begin{frame}{Matrices cuadradas}
\protect\hypertarget{matrices-cuadradas-12}{}
Matriz singular. \(A\) es singular si no tiene inversa, es decir, cuando
no es regular.

\textbf{Ejemplo 35}

\[A = \begin{pmatrix}1&2&3\\4&5&6\\7&8&9\end{pmatrix}\] es una matriz
singular. Más adelante, cuando hablemos de determinantes, veremos el por
qué
\end{frame}

\begin{frame}{Matrices cuadradas}
\protect\hypertarget{matrices-cuadradas-13}{}
Matriz ortogonal. \(A\) es ortogonal si es regular y además su inversa
coincide con su transpuesta. Dicho de otra manera

\[A\text{ ortogonal }\Leftrightarrow AA^t=A^tA = I_n\]

\textbf{Ejemplo 36}

\[A = \frac{1}{3}\begin{pmatrix}2&-2&1\\2&1&-2\\1&2&2\end{pmatrix}\] es
una matriz ortogonal ya que

\[AA^t=\frac{1}{3}\begin{pmatrix}2&-2&1\\2&1&-2\\1&2&2\end{pmatrix}\frac{1}{3}\begin{pmatrix}2&2&1\\-2&1&2\\1&-2&2\end{pmatrix}=\frac{1}{9}\begin{pmatrix}9&0&0\\0&9&0\\0&0&9\end{pmatrix}=I_3\]

\textbf{Ejercicio 22.} Comprobad que \(A^tA=I_3\)
\end{frame}

\begin{frame}{Matrices cuadradas}
\protect\hypertarget{matrices-cuadradas-14}{}
Proposición. Sean \(A,B\in\mathcal{M}_n(\mathbb{K})\). Entonces si \(A\)
y \(B\) son invertibles, también lo es su producto y se cumple:

\[(AB)^{-1}=B^{-1}A^{-1}\]

\textbf{Ejercicio 23.} Sean
\[A = \begin{pmatrix}1&2\\7&8\end{pmatrix}\qquad B=\begin{pmatrix}1&3\\-1&0\end{pmatrix}\]
de donde
\[A^{-1} = -\frac{1}{6}\begin{pmatrix}8&-2\\-7&1\end{pmatrix}\qquad B^{-1}=\frac{1}{3}\begin{pmatrix}0&-3\\1&1\end{pmatrix}\]
Comprobad que \(B^{-1}A^{-1}=(AB)^{-1}\) donde
\((AB)^{-1}=-\frac{1}{18}\begin{pmatrix}21&-3\\1&-1\end{pmatrix}\)
\end{frame}

\begin{frame}{Matrices cuadradas}
\protect\hypertarget{matrices-cuadradas-15}{}
\textbf{Demostración}

Para probar \((AB)^{-1}=B^{-1}A^{-1}\), lo que haremos será ver que
\[(AB)(B^{-1}A^{-1}) = (B^{-1}A^{-1})(AB) = I_n\]

Por un lado, por la propiedad asociativa y como \(B^{-1}\) es la inversa
de \(B\), tenemos

\[(AB)(B^{-1}A^{-1}) = A(BB^{-1})A^{-1} = AI_nA^{-1} = AA^{-1}\] ya que,
recordemos, la matriz identidad \(I_n\) actúa como elemento neutro del
producto de matrices. Ahora, como \(A^{-1}\) es la inversa de \(A\), se
tiene que

\[(AB)(B^{-1}A^{-1}) = AA^{-1} = I_n\] tal y como queríamos demostrar.

\textbf{Ejercicio 24.} Acabar la demostración siguiendo como modelo la
parte que se ha llevado a cabo hasta el momento.
\end{frame}

\hypertarget{resumen}{%
\section{Resumen}\label{resumen}}

\begin{frame}{Resumen}
\protect\hypertarget{resumen-1}{}
Las operaciones anteriores conforman el llamado álgebra matricial. Este
nombre es adecuado ya que gracias a ellas es posible realizar la
manipulación habitual de ecuaciones con matrices igual que se hace con
los números reales siempre y cuando se tenga precaución con aquellas
propiedades que no se verifican, vistas todas ellas anteriormente.

Por ejemplo, en una ecuación con matrices todo lo que esté sumando pasa
al otro término restando y viceversa.

De esta manera, se pueden resolver ecuaciones del tipo: encuentre una
matriz \(X\) tal que \(A+\lambda X=\mu B\) donde \(A\) y \(B\) son
matrices conocidas y \(\lambda\ne 0\) y \(\mu\) son valores de
\(\mathbb{K}\) también conocidos. La solución será
\(X=\frac{1}{\lambda}(\mu B-A)\).
\end{frame}

\begin{frame}{Resumen}
\protect\hypertarget{resumen-2}{}
Nótese sin embargo que las ecuaciones de la forma \(AX=B\) no se pueden
manipular de la forma habitual a no ser que la matriz \(A\) sea cuadrada
e invertible. Entonces se tendrá \(X=A^{-1}B\). Nótese que valdría
multiplicar a la izquierda por \(A^{-1}\) pero no valdría hacerlo a la
derecha. Si la ecuación que tiene es de la forma \(XA=B\) entonces, si
\(A\) es invertible, será \(X=BA^{-1}\), multiplicando a la derecha por
\(A^{-1}\).
\end{frame}

\begin{frame}{Resumen}
\protect\hypertarget{resumen-3}{}
Se pueden calcular también las potencias \(n\)-ésimas de las matrices de
la forma habitual \(A^n=A\cdot A\cdot\cdots A\) (\(n\) veces). Nótese
que el binomio de Newton para calcular \((A+B)^n\) solo se verifica en
los casos en que \(A\) y \(B\) conmuten. Por ejemplo

\[(A+B)^2=A^2+AB+BA+B^2\]

Si \(A\) y \(B\) conmutan, entonces \((A+B)^2=A^2+2AB+B^2\)
\end{frame}

\hypertarget{operaciones-elementales.-matrices-escalonadas}{%
\section{Operaciones elementales. Matrices
escalonadas}\label{operaciones-elementales.-matrices-escalonadas}}

\begin{frame}{Matrices escalonadas}
\protect\hypertarget{matrices-escalonadas}{}
Vamos a introducir ahora un tipo especial de matrices triangulares
superiores (inferiores), las llamadas matrices escalonadas por filas
(por columnas).

Matriz escalonada por filas. Una matriz
\(A\in\mathcal{M}_{m\times n}(\mathbb{K})\) es escalonada por filas
cuando se cumplen simultáneamente las dos condiciones siguientes:

\begin{itemize}
\tightlist
\item
  El primer elemento no nulo de cada fila, denominado pivote, está a la
  derecha del pivote de la fila superior
\item
  Las filas nulas están en la parte inferior de la matriz.
\end{itemize}
\end{frame}

\begin{frame}{Matrices escalonadas}
\protect\hypertarget{matrices-escalonadas-1}{}
\textbf{Ejemplo 37}

Estas matrices son escalonadas por filas:

\[\begin{pmatrix}2&1&-1&2\\0&1&-2&1\\0&0&0&2\\0&0&0&0\end{pmatrix}\qquad \begin{pmatrix}3&2&2&5&8\\0&2&-1&9&-3\\0&0&5&3&2\end{pmatrix} \]
\end{frame}

\begin{frame}{Matrices escalonadas}
\protect\hypertarget{matrices-escalonadas-2}{}
Matriz escalonada reducida. Una matriz
\(A\in\mathcal{M}_{m\times n}(\mathbb{K})\) es escalonada reducida por
filas si es escalonada y además cumple los siguientes requisitos:

\begin{itemize}
\tightlist
\item
  Los pivotes son todos 1's.
\item
  Todos los elementos que están en la misma columna del pivote son
  nulos.
\end{itemize}
\end{frame}

\begin{frame}{Matrices escalonadas}
\protect\hypertarget{matrices-escalonadas-3}{}
\textbf{Ejemplo 38}

Estas matrices son escalonadas reducidas por filas:

\[\begin{pmatrix}1&0&0&2\\0&1&0&1\\0&0&1&-1\\0&0&0&0\end{pmatrix}\qquad \begin{pmatrix}1&0&0&5&8\\0&1&0&7&-3\\0&0&1&10&2\end{pmatrix} \]
\end{frame}

\begin{frame}{Matrices escalonadas}
\protect\hypertarget{matrices-escalonadas-4}{}
\textbf{Ejercicio 25}

\begin{itemize}
\item
  Dad definiciones equivalentes para las matrices escalonadas por
  columnas y matrices escalonadas reducidas por columnas.
\item
  Poned dos ejemplos de cada tipo de matriz.
\end{itemize}
\end{frame}

\begin{frame}{Operaciones elementales de una matriz}
\protect\hypertarget{operaciones-elementales-de-una-matriz}{}
Operaciones elementales por filas. Sea
\(A\in\mathcal{M}_{m\times n}(\mathbb{K})\). Las siguientes operaciones
se llaman operaciones elementales por filas de la matriz \(A\):

\begin{itemize}
\tightlist
\item
  Multiplicar una fila por un \(\lambda\in\mathbb{K},\ \lambda\ne 0\).
\item
  Intercambiar dos filas.
\item
  Sumar un múltiplo de una fila a otra.
\end{itemize}

De manera análoga se pueden definir las operaciones elementales por
columnas.
\end{frame}

\begin{frame}{Matrices equivalentes}
\protect\hypertarget{matrices-equivalentes}{}
Matrices equivalentes por filas. Dos matrices
\(A,B\in\mathcal{M}_{m\times n}(\mathbb{K})\) son equivalentes por filas
(por columnas) si una de ellas se puede obtener a partir de la otra
mediante un número finito de operaciones elementales por filas
(columnas).

Teorema

\begin{itemize}
\tightlist
\item
  Toda matriz es equivalente por filas (columnas) a una matriz
  escalonada por filas (columnas).
\item
  Toda matriz es equivalente por filas (columnas) a una única matriz
  escalonada reducida por filas (columnas).
\end{itemize}
\end{frame}

\begin{frame}{Matrices equivalentes}
\protect\hypertarget{matrices-equivalentes-1}{}
La demostración la haremos de manera constructiva. Es decir, hallaremos
un algoritmo (Método de Gauss) para encontrar la matriz escalonada en
este caso.

\textbf{DEMOSTRACIÓN}

Sea \(A=(a_{ij})\in\mathcal{M}_{m\times n}(\mathbb{K})\), entonces
procederemos de la siguiente manera:

\begin{enumerate}
\item
  Si \(a_{11}\ne 0\), se divide la primera fila por \(a_{11}\) y se
  obtiene una matriz equivalente en la que \(a_{11}=1\). Entonces este
  nuevo \(a_{11}\) será el primer pivote. Ahora, se resta a cada fila
  \(i\) la primera fila multiplicada por \(a_{i1}\). Así, la primera
  resta de elementos de la primera columna será 0 y se pasa al punto 4.
\item
  Si \(a_{11}=0\), se busca el primer \(i\) tal que \(a_{i1}\ne 0\).
  Entonces, se intercambian la primera fila y la \(i\) obteniendo una
  matriz equivalente con un nuevo \(a_{11}\ne 0\). A partir de aquí,
  volvemos al punto 1 y repetimos el proceso.
\item
  Si \(a_{i1}=0\) para todo \(i=1,\dots,m\), entonces dejamos esta
  primera columna de ceros y aplicamos el algoritmo del paso 1 a la
  matriz resultante de eliminar la primera columna.
\item
  Se repite el proceso a la matriz obtenida de eliminar la primera fila
  y la primera columna de nuestra matriz.
\end{enumerate}
\end{frame}

\begin{frame}{Matrices equivalentes}
\protect\hypertarget{matrices-equivalentes-2}{}
Nótese que con el método de la demostración se obtiene la única matriz
escalonada equivalente por filas cuyos pivotes son todos unos.

Para obtener la matriz escalonada reducida, si hay algún elemento
\(a_{ij}\) distinto de cero por encima de algún determinado pivote, se
resta a la fila de este elemento (la fila \(i\)), la fila del pivote
multiplicada por \(a_{ij}\).

Se repite el paso anterior tantas veces como sea necesario y se llega
así a la matriz escalonada reducida equivalente.
\end{frame}

\begin{frame}{Matrices equivalentes}
\protect\hypertarget{matrices-equivalentes-3}{}
\textbf{Ejercicio 26}

Considerad la matriz \(A\in\mathcal{M}_{3\times 4}(\mathbb{R})\) dada
por

\[A=\begin{pmatrix}1&1&3&5\\2&4&3&-2\\-2&2&-1&3\end{pmatrix}\]

Calculad su matriz escalonada y su escalonada reducida por filas.
\end{frame}

\begin{frame}{Matrices equivalentes}
\protect\hypertarget{matrices-equivalentes-4}{}
\textbf{Ejercicio 27}

Considerad la matriz \(A\in\mathcal{M}_{3\times 5}(\mathbb{R})\) dada
por

\[A=\begin{pmatrix}0&0&3&5&-1\\1&0&4&-1&2\\2&3&0&-2&3\end{pmatrix}\]

Calculad su matriz escalonada y su escalonada reducida por filas.
\end{frame}

\hypertarget{rango-de-una-matriz}{%
\section{Rango de una matriz}\label{rango-de-una-matriz}}

\begin{frame}{Rango de una matriz}
\protect\hypertarget{rango-de-una-matriz-1}{}
Dada la unicidad de la matriz escalonada reducida, se pueden definir
conceptos sobre una matriz \(A\) mediante su matriz escalonada reducida
por filas (por columnas) equivalente

Rango. Sea \(A\in\mathcal{M}_{m\times n}(\mathbb{K})\). Se denomina
rango de \(A\) y se denota como \(\text{rg}(A)\), al número de filas no
nulas que tiene su única matriz escalonada (o su escalonada reducida)
por filas equivalentes.

\textbf{Ejemplo 38}

La matriz
\[A = \begin{pmatrix}1&2&2&5&8\\0&1&-1&7&-3\\0&0&1&10&2\end{pmatrix}\]
tiene rango 3
\end{frame}

\begin{frame}{Rango de una matriz}
\protect\hypertarget{rango-de-una-matriz-2}{}
Teorema. Sea \(A\in\mathcal{M}_{m\times n}(\mathbb{K})\). El rango de
\(A\) coincidirá con el número de columnas no nulas de su única matriz
escalonada reducida por columnas equivalente.

En realidad, el número de filas (columnas) no nulas es siempre el mismo
en cualquier matriz equivalente por filas (por columnas) a la dada. Por
tanto, para calcular el rango de una matriz \(A\) bastará con encontrar
una matriz \(B\) escalonada por filas (columnas) equivalente a \(A\) y
contar el número de filas (columnas) no nulas de \(B\).
\end{frame}

\begin{frame}{Rango de una matriz}
\protect\hypertarget{rango-de-una-matriz-3}{}
\textbf{Ejercicio 28}

\begin{itemize}
\item
  Calculad el rango de
  \[A=\begin{pmatrix}1&1&3&5\\2&4&3&-2\\-2&2&-1&3\end{pmatrix}\]
\item
  Calculad el rango de
  \[B = \begin{pmatrix}1&3&1\\0&3&-2\\2&2&3\end{pmatrix}\]
\end{itemize}
\end{frame}

\hypertarget{cuxe1lculo-de-la-matriz-inversa}{%
\section{Cálculo de la matriz
inversa}\label{cuxe1lculo-de-la-matriz-inversa}}

\begin{frame}{Caracterización de las matrices invertibles}
\protect\hypertarget{caracterizaciuxf3n-de-las-matrices-invertibles}{}
Con las matrices escalonadas y las operaciones elementales, no solo se
puede calcular el rango de una matriz sino que también resultan útiles
en el cálculo de matrices inversas como veremos a continuación.

El primer aporte que pueden hacer es la caracterización de las matrices
invertibles a través de su rango y de su matriz escalonada reducida.
\end{frame}

\begin{frame}{Teorema de caracterización}
\protect\hypertarget{teorema-de-caracterizaciuxf3n}{}
Teorema. Sea \(A\) una matriz cuadrada
\(A\in\mathcal{M}_n(\mathbb{K})\). Entonces son equivalentes:

\begin{itemize}
\tightlist
\item
  \(A\) es invertible
\item
  \(rg(A)=n\)
\item
  La matriz escalonada reducida por filas (por columnas) equivalente a
  \(A\) es la matriz identidad \(I_n\)
\end{itemize}
\end{frame}

\begin{frame}{Teorema de caracterización}
\protect\hypertarget{teorema-de-caracterizaciuxf3n-1}{}
Además, la tercera equivalencia aporta un método para calcular la matriz
inversa de una matriz invertible \(A\in\mathcal{M}_n(\mathbb{K})\): Este
consiste en escribir la matriz identidad \(I_n\) a la derecha de la
matriz (escrito de forma abreviada \((A|I_n)\)) y a través de
transformaciones elementales por filas (o por columnas), calcular la
matriz escalonada reducida que será de la forma \((I_n|B)\). La matriz
\(B\) resultante es precisamente la matriz inversa de \(A\), es decir
\(A^{-1}=B\).
\end{frame}

\begin{frame}{Cálculo de la matriz inversa}
\protect\hypertarget{cuxe1lculo-de-la-matriz-inversa-1}{}
\textbf{Ejercicio 29}

Sea \(A\) la matriz cuadrada \(A\in\mathcal{M}_n(\mathbb{K})\) dada por

\[\begin{pmatrix}1&3&-1\\0&2&3\\ -1&0&2\end{pmatrix}\]

Razonad si \(A\) es invertible y, si lo es, calculad su inversa.
\end{frame}

\hypertarget{aplicaciones-de-las-matrices}{%
\section{Aplicaciones de las
matrices}\label{aplicaciones-de-las-matrices}}

\begin{frame}{Aplicaciones de las matrices}
\protect\hypertarget{aplicaciones-de-las-matrices-1}{}
\begin{itemize}
\tightlist
\item
  Álgebra lineal y geometría
\item
  Modelos lineales en ingeniería y economía
\item
  Ecuaciones en diferencias
\item
  Tratamiento de imágenes y diseño asistido por ordenador
\item
  Matrices booleanas, grafos y relaciones
\item
  Matrices estocásticas y estadística
\item
  Ecuaciones diferenciales y sistemas dinámicos
\end{itemize}
\end{frame}

\end{document}
